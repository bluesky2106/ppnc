\documentclass[conference]{IEEEtran}
\IEEEoverridecommandlockouts
% The preceding line is only needed to identify funding in the first footnote. If that is unneeded, please comment it out.
\usepackage{cite}
%\usepackage{amsmath,amssymb,amsfonts}
%\usepackage{algorithmic}
%\usepackage{graphicx}
%\usepackage{textcomp}
%\usepackage{xcolor}
%\usepackage{hyperref}
%\usepackage{multirow}
%\usepackage{colortbl}
\usepackage[english,vietnamese]{babel}

\begin{document}

\title{Các mô hình tiền huấn luyện trong xử lý ngôn ngữ tự nhiên}

\author{
\IEEEauthorblockN{Loc PV. Nguyen}
\IEEEauthorblockA{
\textit{FPT University Global Education} \\
Ho Chi Minh City, Vietnam \\
loc20mse23026@fsb.edu.vn}
\and

\IEEEauthorblockN{Khoi XM. Nguyen}
\IEEEauthorblockA{
\textit{FPT University Global Education} \\
Ho Chi Minh City, Vietnam \\
khoi20mse23024@fsb.edu.vn}
\and

\IEEEauthorblockN{Phuong H. Nguyen}
\IEEEauthorblockA{
\textit{FPT University Global Education} \\
Ho Chi Minh City, Vietnam \\
phuong20mse23020@fsb.edu.vn}
\and

\IEEEauthorblockN{Hoang N. Dang}
\IEEEauthorblockA{
\textit{FPT University Global Education} \\
Ho Chi Minh City, Vietnam \\
hoang20mse23030@fsb.edu.vn}
}

\maketitle

\begin{abstract}
Trong bài khảo sát này chúng tôi sẽ cung cấp cái nhìn toàn diện về thuật ngữ Tiền Mô Hình âĐào Tạo trong Xử Lý Ngôn Ngữ Tự Nhiên (NLP). Mục tiêu của bài khảo sát này là tìm hiểu, nghiên cứu và so sánh tính hiệu quả của kỹ thuật này so với những phương pháp trước kia dựa trên những quy tắc chung.     
\end{abstract}

\begin{IEEEkeywords}
Xử lý ngôn ngữ tự nhiên, Mô hình tiền huấn luyện, Trí tuệ nhân tạo, Học máy.
\end{IEEEkeywords}

\section{Giới thiệu}
Phương pháp Học Chuyển Tiếp (Transfer Learning) là một phương pháp phổ biến trong thị giác máy tính cũng như xử lý ngôn ngữ tự nhiên và nhiều ứng dụng học máy khác. Học chuyển tiếp là một cách tiếp cận trong học sâu (và học máy), nơi kiến thức được chuyển giao từ mô hình này sang mô hình khác.

Với phương pháp học chuyển tiếp, thay vì bắt đầu quá trình huấn luyện (Training) từ đầu, ta có thể bắt đầu học từ các mô hình tiền huấn luyện (Pre-trained model) đã đạt được khi giải quyết một vấn đề khác. Bằng cách này, ta có thể tận dụng những đặc trưng (features) đã học trước đó và tránh bắt đầu lại từ đầu.

Mô hình tiền huấn luyện (Pre-trained model) là một mô hình đã được đào tạo trên một tập dữ liệu chuẩn và đủ lớn để giải quyết một vấn đề tương tự như vấn đề mà chúng ta muốn giải quyết (như xử lý ngôn ngữ tự nhiên..). Do chi phí để huấn luyện các model rất tốn kém, nên thông thường người ta sẽ sử dụng các model từ các nguồn đã được public trước đó (ví dụ: BERT, PhoBERT, Underthesea, VGG, Inception, MobileNet,…).

Với sự phát triển của học sâu, các mạng nơ-ron khác nhau đã được sử dụng rộng rãi để giải quyết các bài toán NLP, chẳng hạn như mạng nơ-ron tích chập (Convolutional Neural Network) ~\cite{gehring2017convolutional, kalchbrenner-etal-2014-convolutional, kim-2014-convolutional}, mạng nơ-ron hồi quy (Recurrent Neural Network) ~\cite{sutskever-2014-sequence, liu-2016-recurrent}, mạng nơ-ron đồ thị (Graph Neural Network) ~\cite{socher-2013-recursivedeep, tai2015improved, marcheggiani2018exploiting}. Các phương pháp NLP không sử dụng mạng nơ-ron thường chủ yếu dựa vào các tính năng được tạo thủ công rời rạc, trong khi các phương pháp thần kinh thường sử dụng các vectơ có chiều thấp và dày đặc (hay còn gọi là biểu diễn phân phối) để thể hiện ngầm định các đặc điểm ngữ nghĩa cú pháp của ngôn ngữ. Những đại diện này được học trong các nhiệm vụ NLP cụ thể. Do đó, các phương pháp thần kinh giúp mọi người dễ dàng phát triển các hệ thống NLP khác nhau.

Gần đây,  việc tiền huấn luyện một Mô hình từ bộ dữ liệu đa dạng ngày càng trở nên phổ biến.  Một cách lý tưởng,  việc tiền huấn luyện (pre-training) này trang bị cho Mô hình các "khả năng" thông dụng cũng như các "kiến thức" để từ đó có thể dùng để chuyển giao cho các tác vụ cụ thể. Trong các ứng dụng của học chuyển giao vào lĩnh vực Thị Giác Máy Tính ~\cite{oquab2014learning, thrun2004advances, minyoung2016ImageNet}, tiền huấn luyện thường được thực hiện thông qua học có giám sát trên một tập dữ liệu lớn-có gán nhãn- như ImageNet ~\cite{Deng2009ImageNet, Russakovsky2015ImageNet}. Ở hướng ngược lại,  các  kỹ thuật hiện đại dùng cho học chuyển giao trong lĩnh vực Xử Lý Ngôn Ngữ Tự Nhiên lại thường thông qua  phương pháp học không giám sát trên bộ dữ liệu không-gán-nhãn.  Cách tiếp cận này gần đây đã được sử dụng để thu được các kết quả tiên tiến trong hầu hết các chỉ số NLP phổ biến ~\cite{kentonbert, dong2019unified,  liu2019roberta}.  Bên cạnh điểm mạnh về thực nghiệm,  tiền huấn luyện không giám sát cho NLP đặc biệt hấp dẫn vì dữ liệu văn bản không gắn nhãn có sẵn rất nhiều trên Internet - ví dụ: Dự án Common Crawl với khoảng 20TB dữ liệu văn bản được trích xuất từ các trang web mỗi tháng.  Điều này, một cách tự nhiên lại rất phù hợp với các mạng nơ-ron,  vốn đã thể hiện khả năng mở rộng đáng kể,  nghĩa là hiệu suất có thể đẩy lên cao  hơn chỉ đơn giản bằng cách đào tạo một mô hình lớn hơn trên tập dữ liệu lớn hơn ~\cite{hestnessdeep, shazeer2017outrageously, jozefowicz2016exploring, mahajan2018exploring, radford2019language}

\bibliographystyle{IEEEtran}
\bibliography{bibfile}

\end{document}